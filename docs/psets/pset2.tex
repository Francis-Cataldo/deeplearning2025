\documentclass{article}
\usepackage[utf8]{inputenc}
\usepackage[a4paper, total={6in, 8in}]{geometry}
\usepackage{amsmath, amsfonts}
\usepackage{hyperref,bbm, graphicx}

\DeclareMathOperator*{\argmin}{arg\,min}

\title{CSCI 1051 Problem Set 2}
\author{} % TODO: Put your name here
\date{\today}

\begin{document}

\maketitle

\subsection*{Submission Instructions}

Please upload your solutions by
\textbf{5pm Friday January 17, 2025.}
\begin{itemize}
\item You are encouraged to discuss ideas
and work with your classmates. However, you
\textbf{must acknowledge} your collaborators
at the top of each solution on which
you collaborated with others 
and you \textbf{must write} your solutions
independently.
\item Your solutions to theory questions must
be written legibly, or typeset in LaTeX or markdown.
If you would like to use LaTeX, you can import the source of this document 
\href{https://www.rtealwitter.com/deeplearning/psets/pset2.tex}{here}
to Overleaf.
\item I recommend that you write your solutions to coding questions in a Jupyter notebook using Google Colab.
\item You should submit your solutions as a \textbf{single PDF} via the assignment on Gradescope.
\end{itemize}

\newpage \section*{Problem 1: Visualizing Word Embeddings}

In this problem, we will embed words using a pretrained embedding model. I recommend using the Pytorch \texttt{AutoTokenizer} and \texttt{AutoModel} with the ``bert-base-uncased'' pretrained weights.

\subsection*{Part A: 10D Visualizations}

Embed (at least 8) words of your choice then create a visualization of the first ten embedding dimensions for each word.

\subsection*{Part B: 2D Plots}

Plot the words in the first two embedding dimensions on a 2D plot.
Do you notice any patterns?

\subsection*{Part C: 2D Plots via PCA}

Use PCA to the embeddings and plot the resulting two principal components on a 2D plot.
Do you notice any patterns?

\subsection*{Part D: Word Math}

Try adding and subtracting word embeddings from each other, then plot the results in the 2D plots from Part B and Part C, respectively.
Is the word you create close to what you imagined it would be?

For example, we may expect that
\begin{align*}
    \textnormal{king} - \textnormal{man} + \textnormal{woman} \approx \textnormal{queen}.
\end{align*}

%\input{solutions/solution2_1}

\newpage
\section*{Problem 2: Visualizing Attention}

\subsection*{Part A: Self-attention Activations}

Consider a pretrained transformer model.
Select a text input of your choice.
Plot the self-attention weights at several layers (both early and later) in your transformer model. What patterns do you notice?

\subsection*{Part B: Cross-attention Activations}

Consider a pretrained transformer model for translation.
Select a text input of your choice and translate it.
Now feed the original and translated text to a translation model and plot the cross-attention weights at several layers (both early and later). What patterns do you notice?

%\input{solutions/solution2_2}

\newpage
\section*{Problem 3: Low Rank Adaptation}

In this problem, we will investigate the advantages and disadvantages of low rank adaptation.

\subsection*{Part A: Architectures}

Build a neural network with large linear layers ($\geq$ 1 million parameters).
In addition, build a similar neural network architecture where each layer $d_\textnormal{in} \times d_\textnormal{out}$ is replaced by two linear layers of dimensions $d_\textnormal{in} \times r$ and $r \times d_\textnormal{out}$ for a small rank $r \approx 100$.

\subsection*{Part B: Comparison}

Train instances of both models on a dataset of your choice.
Compare the number of parameters in each model and the time it take to train each model.
Finally, plot the test loss by epoch for the two models.

%\input{solutions/solution2_3}

\end{document}